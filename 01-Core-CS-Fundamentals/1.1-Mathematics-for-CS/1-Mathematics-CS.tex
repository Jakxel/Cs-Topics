\documentclass[11pt]{article}
\usepackage[utf8]{inputenc}
\usepackage[T1]{fontenc}
\usepackage[spanish]{babel}
\usepackage{amsmath, amssymb, amsthm}
\usepackage{hyperref}
\usepackage{enumitem}
\usepackage{graphicx}
\usepackage{xcolor}
\usepackage{fancyhdr}
\usepackage{tcolorbox}
\usepackage{geometry}
\usepackage{listings}

\geometry{a4paper, margin=2.5cm}

% Encabezado
\pagestyle{fancy}
\fancyhf{}
\rhead{Mathematics for CS}
\lhead{Jakxel}
\cfoot{\thepage}

% Estilos para definiciones y demás
\newtheorem{definition}{Definición}[section]
\newtheorem{theorem}{Teorema}[section]
\newtheorem{example}{Ejemplo}[section]

\newtcolorbox{keyformula}{
  colback=blue!5!white,
  colframe=blue!80!black,
  title=Fórmula clave,
  fonttitle=\bfseries,
  sharp corners,
  before skip=10pt, after skip=10pt
}

% Configuración de listings para código
\lstset{
  language=C++,
  basicstyle=\ttfamily\small,
  keywordstyle=\color{blue},
  stringstyle=\color{teal},
  commentstyle=\color{gray},
  numbers=left,
  numberstyle=\tiny,
  stepnumber=1,
  breaklines=true,
  frame=single,
  tabsize=2,
  showstringspaces=false
}

\title{Mathematics for Computer Science}
\author{Jakxel}
\date{\today}

\begin{document}

\maketitle
\tableofcontents
\newpage

% ============================
\section{Lógica Proposicional}
\subsection{Conectivos Lógicos}
\begin{definition}
Un \textbf{conectivo lógico} es un operador que une dos o más proposiciones para formar una proposición compuesta. Ejemplos comunes: $\neg$, $\land$, $\lor$, $\rightarrow$, $\leftrightarrow$.
\end{definition}

\begin{keyformula}
$p \rightarrow q \equiv \neg p \lor q$
\end{keyformula}

\subsection{Tablas de verdad}
\begin{example}
Verifica la validez de la siguiente proposición: $((p \rightarrow q) \land (q \rightarrow r)) \rightarrow (p \rightarrow r)$
\end{example}

% ============================
\section{Teoría de Conjuntos}
\subsection{Operaciones}
\begin{definition}
La \textbf{intersección} de dos conjuntos $A$ y $B$ es el conjunto de todos los elementos que pertenecen a ambos: $A \cap B$.
\end{definition}

\begin{keyformula}
\[
\neg (A \cup B) = \neg A \cap \neg B \quad \text{y} \quad \neg (A \cap B) = \neg A \cup \neg B
\]
\end{keyformula}

% ============================
\section{Teoría de Números}
\subsection{Congruencias}
\begin{definition}
$a \equiv b \pmod{n}$ significa que $n$ divide a $a - b$.
\end{definition}

\begin{theorem}[Pequeño Teorema de Fermat]
Si $p$ es un número primo y $a$ no es divisible por $p$, entonces:
\[
a^{p-1} \equiv 1 \pmod{p}
\]
\end{theorem}

% ============================
\section{Combinatoria}
\subsection{Permutaciones y Combinaciones}
\begin{keyformula}
Permutaciones sin repetición: $P(n, r) = \frac{n!}{(n - r)!}$
\end{keyformula}

\begin{keyformula}
Combinaciones: $C(n, r) = \binom{n}{r} = \frac{n!}{r!(n - r)!}$
\end{keyformula}

% ============================
\section{Probabilidad discreta 1}
\subsection{Eventos}
\begin{definition}
Un \textbf{evento} es un subconjunto del espacio muestral. Por ejemplo, al lanzar un dado, obtener un número par es un evento: $\{2, 4, 6\}$.
\end{definition}

\begin{example}
¿Cuál es la probabilidad de obtener al menos una cara al lanzar dos monedas?
\end{example}

% ============================
\section{Grafos}
\subsection{Definición}
\begin{definition}
Un \textbf{grafo} $G = (V, E)$ consiste en un conjunto de vértices $V$ y un conjunto de aristas $E \subseteq V \times V$.
\end{definition}

\subsection{Árboles}
\begin{theorem}
Todo árbol con $n$ vértices tiene exactamente $n - 1$ aristas.
\end{theorem}

% ============================
\section{Código en C++}
\subsection{Ejemplo: Factorial Recursivo}

\begin{definition}
En muchas estructuras algorítmicas, es común definir funciones recursivas para resolver problemas. Aquí mostramos un ejemplo del cálculo del factorial usando C++.
\end{definition}

\begin{lstlisting}
// C++: Factorial recursivo
#include <iostream>
using namespace std;

int factorial(int n) {
  if (n <= 1) return 1;
  return n * factorial(n - 1);
}

int main() {
  int n = 5;
  cout << "Factorial de " << n << " es: " << factorial(n) << endl;
  return 0;
}
\end{lstlisting}

\begin{example}
El código anterior imprimirá: \texttt{Factorial de 5 es: 120}
\end{example}

\subsection{Código desde archivo externo}
\lstinputlisting[language=C++]{factorial.cpp}

hola espero que esten este una prueba de live server con latex
\end{document}
