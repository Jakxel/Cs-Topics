\documentclass[11pt]{article}
\usepackage[utf8]{inputenc}
\usepackage[T1]{fontenc}
\usepackage[spanish,english]{babel}
\usepackage{geometry}
\usepackage{hyperref}
\usepackage{lmodern}

\geometry{margin=2.5cm}

\begin{document}

\begin{center}
    {\Huge \textbf{Computer Networks Notes}}\\[0.5em]
    {\large Victor Jakxel Islas Carreon}
\end{center}

\newpage
\tableofcontents
\newpage

\section*{1. Network Overview \\ \normalsize{\textit{pubDate: 2025-05-01}}}

\subsection*{What is a Network?}
A \textbf{network} is a group of interconnected devices that can communicate and share resources. Think of it as a system where computers and other devices exchange information, collaborate, and access shared resources like printers or storage.

Computers use common \textbf{communication protocols} over \textbf{digital interconnections} to communicate with each other. These interconnections are built using \textbf{telecommunications network technologies}, which may be based on physical wiring, optical fiber, or wireless radio-frequency methods. These can be arranged in a variety of \textbf{network topologies}.

\subsection*{Network Packet}
Most modern computer networks use protocols based on \textbf{packet-mode transmission}. A \textbf{network packet} is a formatted unit of data carried by a packet-switched network.

A packet consists of two types of data:
\begin{itemize}
    \item \textbf{Control information}
    \item \textbf{User data} (also called the \textbf{payload})
\end{itemize}

The control information provides the network with the necessary data to deliver the user payload correctly. This includes things like:
\begin{itemize}
    \item Source and destination network addresses
    \item Error detection codes
    \item Sequencing information
\end{itemize}

Typically, the control information is found in the \textbf{packet header}, while the payload is carried in the body of the packet.

\subsection*{Types of Networks}
Key examples include:
\begin{itemize}
    \item \textbf{Local Area Networks (LANs)}
    \item \textbf{Wide Area Networks (WANs)}
    \item \textbf{Personal Area Networks (PANs)}
    \item \textbf{Metropolitan Area Networks (MANs)}
    \item \textbf{Campus Area Networks (CANs)}
    \item \textbf{Virtual Private Networks (VPNs)}
\end{itemize}

\subsection*{LAN}
A LAN connects computers and devices within a \textbf{limited area}, such as a home, office, or building.

\subsection*{MAN}
A MAN covers a \textbf{larger area than a LAN}, such as a city or a large campus.

\subsection*{WAN}
A WAN spans a \textbf{very large geographical area}, such as multiple cities or continents.

\subsection*{PAN}
A PAN is the \textbf{smallest and simplest type of network}, used to connect personal devices over a short range.

\subsection*{Network Topology}
\textbf{Network topology} refers to the arrangement of different elements such as nodes, links, or devices in a computer network.

\subsection*{What is Network Topology?}
There are two major categories:
\begin{itemize}
    \item \textbf{Physical Topology}
    \item \textbf{Logical Topology}
\end{itemize}

\subsection*{Types of Topologies}
\begin{itemize}
    \item \textbf{Point-to-Point}
    \item \textbf{Mesh Topology}
    \item \textbf{Bus Topology}
    \item \textbf{Star Topology}
    \item \textbf{Ring Topology}
    \item \textbf{Tree Topology}
    \item \textbf{Hybrid Topology}
\end{itemize}

\subsection*{Client-Server vs Peer-to-Peer Models}
\textbf{Client-server model:} A computing model where a server hosts, manages, and delivers resources and services to one or more clients.

\textbf{Peer-to-peer model:} A decentralized network where participants (peers) interact directly with each other, sharing resources.

\begin{center}
\begin{tabular}{|p{4cm}|p{5cm}|p{5cm}|}
\hline
\textbf{Feature} & \textbf{Client-Server Model} & \textbf{Peer-to-Peer Model} \\
\hline
Architecture & Centralized & Decentralized \\
\hline
Resource Management & Managed by the server & Shared by all peers \\
\hline
Scalability & Highly scalable & Limited by peers \\
\hline
Examples & Web services, email systems, databases & BitTorrent, blockchain \\
\hline
Dependency & Relies on central server & No central dependency \\
\hline
Maintenance & Easier to manage & Harder to manage \\
\hline
\end{tabular}
\end{center}

\newpage
% =====================
\section*{2. Layered Networking Architecture \\ \normalsize{\textit{pubDate: 2025-05-02}}}

\subsection*{1. Concept of layering}
KLayering in networking is the practice of breaking down complex communication systems into a series of \textbf{smaller, manageable, and well-defined layers}.

\textbf{Key Characteristics:}
\begin{itemize}
    \item each layer is responsible for a \textbf{speific Aspect}
    \item layers \textbf{only communicate} with the layer above or below (encapsulation)
    \item Changes in one layer can be made \textbf{independently} (modularity)
\end{itemize}

\subsection*{Why networking is divided into layers?}
\begin{enumerate}
    \item \textbf{Modularity}
    \item \textbf{Interoperality}
    \item \textbf{Simplified Troubleshooting}
    \item \textbf{Flexibility and Upgradability}
    \item \textbf{Standardization}
\end{enumerate}

\subsection*{Real-World Analogy}
Think of sending a package:
\begin{itemize}
    \item You write the letter (Application Layer)
    \item Put it in an envelope (Presentation Layer)
    \item Add sender/receiver info (Session Layer)
    \item Hand it to a delivery service (Transport Layer)
    \item The package is routed (Network Layer)
    \item Carried via trucks or planes (Data Link/Physical Layers)
\end{itemize}

\subsection*{What is an OSI/TCP-IP model?}
OSI is a more general theoretical model with seven layers. TCP/IP is practical, used on the internet, with four layers.

\subsection*{OSI model}
\begin{itemize}
    \item 7. Application layer
    \item 6. Presentation layer
    \item 5. Session layer
    \item 4. Transport layer
    \item 3. Network layer
    \item 2. Data link layer
    \item 1. Physical layer
\end{itemize}

\subsection*{TCP-IP model}
\begin{itemize}
    \item 4. Application layer
    \item 3. Transport layer
    \item 2. Internet layer
    \item 1. Data link / Physical layer
\end{itemize}

\newpage
% =====================
\section*{3. Network Hardware and Transmission \\ \normalsize{\textit{pubDate: 2025-05-16}}}

\subsection*{Network Interface Cards (NICs)}
A Network Interface Card (NIC) is a hardware component that connects a computer to a network, either through cables (wired) or wirelessly (Wi-Fi).  
Inside the NIC, there is a specialized integrated circuit called the network interface controller.

\subsection*{Hubs, switches, routers, access points}
\textbf{What is a hub?}  
Broadcasts data to all connected devices.

\textbf{What is a switch?}  
Sends data only to the intended recipient.

\textbf{What is a router?}  
Connects different networks and often provides Wi-Fi access.

\subsection*{The Access Point}
The access point is a device that interconnects wireless communication devices, forming a wireless network. It allows client machines to connect without cables and without bandwidth limitation.

\subsection*{Transmission Media (wired/wireless)}
A transmission media is the physical medium through which data is transmitted.

\textbf{Guided media:}
\begin{itemize}
    \item Twisted pair cable
    \item Coaxial cable
    \item Optical fiber cable
    \item Stripline
\end{itemize}

\textbf{Unguided media:}
\begin{itemize}
    \item Radio waves
    \item Microwaves
    \item Infrared
\end{itemize}

\subsection*{Signal encoding and modulation}
Encoding transforms data into a suitable format for transmission, while modulation adapts a carrier signal to carry the encoded data.

\end{document}
